%!TEX root = ../main.tex

\section{Introduction}\label{sec:intro}

\begin{comment}
\subsection{Contribution.}
The salient contributions of this work are as follows:
\begin{itemize}
\item We propose a joint model for entity disambiguation and quantity fact extraction in quantitative relational web tables.
\item We introduce TabQs, a system for answering quantity queries from web tables.
\item  We present extensive experiments on real-world data, showing the high quality of our approach.
\item A demo of our search system can be found at: \\ \textcolor{blue}{\url{http://tabqs-demo.webredirect.org/table/index.html}}
\end{itemize}
\end{comment}


%together with title, header, abstract should ideally end at end of first page 
%(with some leeway for spill-over)

\noindent{\bf Motivation.}
%
%
%quantity-centric queries, beyond lookups
%filters, comparisons, aggregates/rankings
%here: focus on quantity filters, as a building block
%give three examples
%simple
%- asian buildings/towers above/taller than 400 meters
%finance
%- football clubs worth more than 2 billion
%technology
%- mobile phones with battery capacity above 5000 mAh
%or tablets with ...
%- electric cars with energy consumption above 80 MPGe
%sports
%- sprinters who ran 100 meters under 9.9 seconds
%
%
A good fraction of web queries revolve around quantities of
entities: looking up, filtering, comparing and aggregating
quantitative properties such as heights of buildings, 
running times of athletes, goals or scoring rates of footballers,
energy consumption of electric cars, etc. 
\cite{DBLP:conf/semweb/HoIPBW19,DBLP:conf/wsdm/BondarenkoBVAFP20,DBLP:journals/corr/abs-2001-03272}.
In this paper we focus on {\em quantity filters}
\cite{DBLP:conf/semweb/HoIPBW19,DBLP:conf/wsdm/HoPKBW20}, 
%which are 
an important class
of queries and also a building block for comparative search.
Examples are:
\squishlist
%\item skyscrapers in Asia taller than 400 meters,
\item British football teams worth more than 1.5 billion pounds
\item sprinters who ran 100 m under 9.9 seconds
\item electric cars with energy efficiency above 80 MPG-e
\squishend


%point out that these are very different from lookups
%and that search engines and QA assistants don't handle them
Note that these are more difficult than quantity lookups,
such as 
%``the height of the Shanghai tower'' 
``the value of Manchester City''
or ``the 
personal 100 m record of Usain Bolt''.
Lookups are well supported by search engines and QA assistants.
Quantity filters, on the other hand, lack this support 
as conditions like ``more than 1.5 billion pounds'' or
``under 9.9 seconds'' are mostly interpreted in
string-matching mode. 
For some examples, search engines return good
web pages, such as Wikipedia articles on ``10-second barrier''
or ``100 metres'', but this is not the user's query intent
and she has to tediously sift through these pages rather
than receiving a crisp entity-list answer. 
Moreover, the result quality depends on the value in the query,
as some (string-interpreted) values match good list pages. For example, there is a list
of 100m races under 10 seconds, but none ready for 9.9, 9.8, etc.


%would be easy if KGs would have this data,
%or if one would know a specific high-quality DB (or web list/table) with the data
%alternatively, extract from a variety of highly heterogeneous sources
Instead of tapping the web, we could turn to knowledge bases (KBs)
and structured sources in the Open Data ecosystem. However, KBs
hardly cover quantities; for example, Wikidata contains thousands of
sprinters but knows their personal records only for a few instances.
To tap Open Data sources, one would still have to find the relevant datasets
in a sea of data sources, and assess their freshness and completeness.

\begin{table}[t]
\centering
\caption[]{
% Illustrative example on f
Football teams.}
\label{table:ExampleTable}
\vspace{\tsq}
{\footnotesize
\begin{tabular}{|l|l|l|l|l|}\hline
Team & Stadium & Capacity & Coach & Value (in Bio) \\ \hline
% Name & Site & Seats & Name & Value
% Team & Site & Seats & Team Mgr & Cost
Bayern & Allianz Arena & ca. 75000 & Hansi Flick &  2.549 Euro \\ 
% \hline
%https://www.forbes.com/teams/bayern-munich/
Real & Bernab\'{e}u & 81,044 & Zidane &  3.649 Euro \\ 
% \hline
%https://www.forbes.com/teams/real-madrid/#46086b336ed4
Man City & unknown & n/a & Pep Guardiola & 2.055 GBP \\ 
% \hline
Chelsea & Stamford Bridge & 40,834 & Frank Lampard & 1.958 GBP  \\ 
% \hline
%https://www.forbes.com/teams/chelsea/#38e1153c3725
Liverpool & Anfield & 53,394 & J\"urgen Klopp & ca. 1.7 GBP    \\ 
\hline
%Liverpool & Anfield & ??? & J\"urgen Klopp & ca. 1.7 GBP    \\ \hline
%https://www.forbes.com/teams/liverpool/#2c8bf1b46045

%https://www.forbes.com/soccer-valuations/list/#tab:overall
\end{tabular}
}%\small
	
\end{table}

%%%%%%%%%%%%%%%%%%%%%%%%%%%%%%%%%%%%%%%%%%%%%%%%%%%%%%%%%%%%%%%%%%%%%%%%
\vspace{0.1cm}
\noindent{\bf Problem.}
%
%quantity fact (fact, for short) extraction
%focus on web tables as input (as text was addressed in prior works)
%
At the core of answering quantity-filter queries is the problem of
extracting entity-quantity facts from web sources.
This has been successfully addressed in \cite{DBLP:conf/semweb/HoIPBW19} for
the case of single sentences from text sources, by recognizing entity-quantity pairs along with relevant context words and
 building on prior work for spotting quantities with numeric values
 and units \cite{DBLP:conf/kdd/SarawagiC14,DBLP:journals/tacl/RoyVR15,DBLP:conf/acl/SahaPM17}.
In this paper, we aim to tap into a different kind of data sources,
namely, ad-hoc {\em web tables} embedded in HTML pages, and address
the problem of accurately extracting entity-quantity facts with
relevant context. 
An {illustrative} example, 
%%%GW: good
which could serve to answer
the  query about British football teams, is shown in
Table \ref{table:ExampleTable}.


%involves the following steps, illustrated by example
%on first glance, prior works on web table extraction seems promising
%but ...
%point out the diffficulties
%
There are good prior works on extracting entity-centric facts from
web tables, including %\cite{DBLP:journals/pvldb/LimayeSC10,DBLP:conf/kdd/SarawagiC14,DBLP:journals/pvldb/VenetisHMPSWMW11,DBLP:journals/pvldb/CafarellaHLMYWW18,DBLP:journals/pvldb/LehmbergB17,DBLP:conf/edbt/OulabiB19,DBLP:conf/wims/LehmbergB19} and a recent survey \cite{DBLP:journals/tist/ZhangB20}. 
surveys 
\cite{DBLP:journals/pvldb/CafarellaHLMYWW18,DBLP:conf/acl/DongHLS20,DBLP:journals/tist/ZhangB20}.
The output is typically a set of
subject-predicate-object (SPO) triples, obtained by 
judiciously
picking two cells 
in the same row as S and O 
and deriving P from the column header of O. 
In conjunction with entity linking to a KG 
%\cite{DBLP:journals/pvldb/LimayeSC10,DBLP:conf/semweb/BhagavatulaND15,DBLP:conf/wims/RitzeLB15,DBLP:conf/cikm/IbrahimRW16,DBLP:conf/edbt/RitzeB17},
\cite{DBLP:journals/tkde/ShenWH15},
an extractor could yield, for instance, 
(\textit{Real Madrid, hasCoach, Zinedine Zidane}).

However, state-of-the-art methods do not work well for quantity facts
for several reasons:
\squishlist
\item First, quantities appear in very diverse and potentially noisy forms.
For example, the team values in Table \ref{table:ExampleTable} 
 are just strings,
varying in units and scale and missing values
(``unknown'', ``n/a'').
%in their units, and scale (billions) in the column header. Also values may be missing
%(unknown or n/a).
%Tables from web sites often pose even more difficult inputs.
Proper interpretation of table cells 
may require understanding the surrounding text.

%
\item Second, it can be hard to infer which column pair denotes a quantity fact,
that is, to which entity column a quantity column refers.
In the example Table \ref{table:ExampleTable}, we need to determine that Capacity refers to
Stadium and Value to Team, but this is not obvious for a machine.
This is further aggravated by the common situation that
column headers are more generic and less informative.
For example, instead of headers like Team, Stadium, Capacity, etc.
we could have Name, Site, Size, etc., which are hard
to interpret.
Prior works on web tables seemed to assume that all columns (for possible choices of O) refer
to the same column (for S), and that this per-row-entity column is usually
the leftmost one \cite{DBLP:journals/pvldb/CafarellaHLMYWW18, DBLP:journals/tist/ZhangB20}. However, these assumptions are not always true.
%
\item Third, extracting entity-quantity pairs alone is not sufficient
for query answering, as many queries include additional modifiers
such as ``British'' or cues for the measure of interest such as
``energy efficiency''. 
To be able to match these against a repository of quantity facts,
the fact extraction needs to capture also relevant context.
Prior works on triples from web tables ignored this important issue; they viewed the extraction as uncoupled from downstream use cases like user queries and questions. 
\squishend


%%%%%%%%%%%%%%%%%%%%%%%%%%%%%%%%%%%%%%%%%%%%%%%%%%%%%%%%%%%%%%%%%%%%%%%
\vspace{0.1cm}
\noindent{\bf Approach.}
%
%one short par, sketching the approach
%
This paper addresses the outlined problems and presents
a full-fledged solution, called {\bf QuTE} (\textbf{Qu}antity \textbf{T}able \textbf{E}xtraction), for extracting contextualized quantity facts
from web tables, to support quantity-filter queries. 
%%% now explain, in a nutshell, how we overcome each of the above three problems
%issue 1
First, to cope with noisy quantities and diverse units and scales in tables, we employ pattern-based extractors and 
rule-based normalization.
%issue 2
Second, for the problem of aligning the right pair
of entity and quantity columns, one of the key tasks, 
we devise a statistical inference method
that leverages external text corpora.
%We also revisit the entity linking, and 
%integrate it with the column alignment for
%joint inference.
%%issue 3
Third, to contextualize the extracted quantity facts,
we exploit text and DOM-tree markup that surround a table, and we introduce a novel way of computing confidence scores for quantity facts, based on evidence in text collections.
%
Finally, as the resulting facts may still yield
many 
false positives in query results,
we have developed additional methods for
enhanced scoring at query time
% , with external evidence from text and corroboration
based on consistency learning
\cite{DBLP:conf/mir/YagnikI07}.



%%%%%%%%%%%%%%%%%%%%%%%%%%%%%%%%%%%%%%%%%%%%%%%%%%%%%%%%%%%%%%%%%%%%%%%
\vspace{0.1cm}
\noindent{\bf Contribution.}
%
%one short par on contributions
%
%Our methodology includes the following novel contributions:
The following are novel contributions:
\squishlist
\item We present 
a robust solution for the 
column alignment problem posed by complex tables,
by harnessing external text corpora
and joint inference with entity linking.
This is the first method specifically geared for
extracting quantity facts, with the novel technique of
leveraging cues from a large text corpus (Sections \ref{sec:columnalignment}
and \ref{sec:qfact_scoring_model}).
%\kp{should we remove the word First here? making it slightly soft}
%\todo{not the first, but the first for quantity??}
%\GW{rephrased this}
\item We introduce a new way of computing quantity fact confidence scores, by incorporating evidence from text collection, with type-based inference to overcome sparseness problems (Section \ref{sec:qfact_scoring_model}).
\item We present a new method for corroborating
extracted facts at query time, re-ranking them and pruning false positives
based on a technique for consistency learning (Section \ref{sec:querying}).
%(originally devised for a different problem \cite{DBLP:conf/mir/YagnikI07}). 
%\cite{DBLP:conf/mir/YagnikI07}.
\item Experiments include comparative evaluations of
our major building blocks against 
%state-the-art 
various
baselines, and an extrinsic study of how well
the extracted facts support quantity queries.
The latter is based on a benchmark of 100 queries
from \cite{DBLP:conf/semweb/HoIPBW19} and a new
collection of 150 queries with list-based ground-truth.
\squishend
%\GW{we need to add the Qfact confidence scoring as a novel contribution, and clearly refer to Sections 4, 4 and 5.2 as our contributions -- this needs some further changes in the intro, will do later or tomorrow!}\\
%\GW{done}
% Experimental data and code are accessible at:\\
% URL
% \footnote{\textcolor{blue}{\url{URL}}}.
% \footnote{\url{URL}}.
% Experimental data, code and 
% a QuTE-based search demonstrator is available at \textcolor{blue}{\url{https://www.mpi-inf.mpg.de/research/quantity-search/quantity-table-extraction/}}
We make the experimental data and code available at 
\footnote{
% \textcolor{blue}{
~\url{https://www.mpi-inf.mpg.de/research/quantity-search/quantity-table-extraction}
% }
}.
A QuTE-based search demonstrator is accessible at
\footnote{
% \textcolor{blue}{
~\url{https://qsearch.mpi-inf.mpg.de/table/}
% }
}.


